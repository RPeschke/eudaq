% !TEX root = EUDAQUserManual.tex
\section{Installing EUDAQ}

\subsection{Prerequisites}
EUDAQ has relatively few dependencies on other software, but some
features do rely on other packages.
To compile EUDAQ from source code, the CMake cross-platform, open-source build system is used.
The libusb library is only needed to communicate over USB with a \gls{TLU}\cite{Cussans2009}.
The VME driver is only needed for reading out \glspl{EUDRB}\cite{Cotta2008}
via VME with a Motorola MVME6100 single board computer.
The other dependencies are only needed for running the DAQ, and not for the common library
(for example if you only want to perform data analysis,
or write a custom Producer to run in the EUDET telescope,
but not run the whole DAQ yourself).

\subsection{CMake}
In order to automatically generate configuration files for the build
process of EUDAQ both compiler and platform independent, the CMake
build system is used.

CMake is available for all major operating systems from
\url{http://www.cmake.org/cmake/resources/software.html}. On most
Linux distributions, it can usually be installed via the built-in
package manager (aptitude/apt-get/yum etc.) and on OSX using
packages provided by e.g. the MacPorts or Fink projects.

\subsubsection{libusb}
In order to communicate with a \gls{TLU}, the libusb library is needed.
Therefore, if you want to compile the \texttt{TLU} subdirectory, you should make sure that libusb is properly installed.

On Mac OS X, this can be installed using Fink or MacPorts.
If using MacPorts you may also need to install the \texttt{libusb-compat} package.
On Linux it may already be installed,
otherwise you should use the built-in package manager to install it.
Make sure to get the development version, which may be named \texttt{libusb-devel} instead of simply \texttt{libusb}.
On Windows, libusb is only needed if compiling with cygwin,
in which case you should use the cygwin installer to install libusb.
Otherwise libusb is not needed, as the included ZestSC1 libraries should work as they are.

\subsubsection{VME driver}
In order to communicate with the \gls{EUDRB} boards a VME library is needed.
A kernel module is included for the Tsi148 VME bridge,
for use on a Motorola MVME6100, in the \texttt{extern/Tsi148} subdirectory.
Installation of this module is beyond the scope of this document.

The \texttt{vme} subdirectory includes code for accessing the VME bus with the Tsi148 module.
In principle other VME bridges could be used,
you just need to write a C++ class that inherits from the VMEInterface class
and implements the necessary methods (look at the TSI148Interface class for an example).

\subsubsection{Qt}
The graphical interface of EUDAQ uses the Qt graphical framework.
In order to compile the \texttt{gui} subdirectory, you must therefore have Qt installed.
It is available in most Linux distributions as the package \texttt{qt4-devel},
but make sure the version is at least 4.4, since there are a few issues with earlier versions.

If the included version is too old, or on other platforms,
it can be downloaded from \url{http://qt.nokia.com/downloads}.
Select the LGPL (free) version, then choose the complete development environment
(it may also work with just the framework, but this is untested).
Make sure the \texttt{QTDIR} environment variable is set to the Qt installation directory,
and the \texttt{\$QTDIR/bin} directory is in your path.

If you are using OSX, the easiest way to install Qt is using the
packages provided by the MacPorts project (\url{http://www.macports.org/}).

\subsubsection{Root}\label{sec:Root}
The online monitor, as well as a few command-line utilities (contained in the \texttt{root} subdirectory),
use the Root package for histogramming.
It can be downloaded from \url{http://root.cern.ch} or installed via
your favorite package manager.
Make sure Root's \texttt{bin} subdirectory is in your path, so that the \texttt{root-config} utility can be run.
This can be done by sourcing the \texttt{thisroot.sh} (or \texttt{thisroot.ch} for csh-like shells)
script in the \texttt{bin} directory of the Root installation:
\begin{listing}[mybash]
source /path/to/root/bin/thisroot.sh
\end{listing}

\subsubsection{LCIO / EUTelescope}\label{sec:LCIO-EUTel}
To enable the writing of \gls{LCIO} files, or the conversion of native files to \gls{LCIO} format,
eudaq must be linked against the \gls{LCIO} and EUTelescope libraries.
Detailed instructions on how to install both using the
\texttt{ilcinstall} scripts can be found at \url{http://eutelescope.web.cern.ch/content/installation}.

The \texttt{EUTELESCOPE} and \texttt{LCIO} environment variables should be set to the
installation directories of EUTelescope and LCIO respectively.
This can be done by sourcing the \texttt{build\_env.sh} script as follows:
\begin{listing}[mybash]
source /path/to/Eutelescope/build_env.sh
\end{listing}

\subsection{Downloading}
The EUDAQ source code is hosted on github. The recommended way to obtain the software is with git,
since this will allow you to easily update to newer versions.
The latest version can be checked out with the following command:
\begin{listing}[mybash]
git clone https://github.com/eudaq/eudaq.git eudaq
\end{listing}

This will create the directory \texttt{eudaq}, and download the latest
development version into it. 
If you already have a copy installed, and want to update it to the
latest version, you do not need to clone the repository again, just change to the \texttt{eudaq} directory use the command:
\begin{listing}[mybash]
git pull
\end{listing}
to update your local copy with all changes commited to the central repository.

Alternatively you can also download a zip file from
\url{https://github.com/eudaq/eudaq/archive/master.zip}.

For production environments (e.g. testbeams) we strongly recommend to
use the latest release version. Use the command \texttt{git tag} in
the repository to find the newest version and type e.g. 
\begin{listing}[mybash]
git checkout tags/v01-01-00
\end{listing}
to change to version 1.1.0.

\subsection{Configuring via CMake}
CMake supports out-of-source configurations and builds -- just enter
the './build' directory and run CMake, i.e.
\begin{listing}[mybash]
cd build
cmake ..
\end{listing}

CMake automatically searches for all required packages and verifies
that all dependencies are met using the \texttt{CMakeLists.txt} script in the
main folder. By default, only the central shared library, the main
executables and (if Qt4 or Qt5 have been found) the graphical user
interface (GUI) are configured for compilation. You can modify this
default behavior by passing the \texttt{BUILD\_[name]} option to
CMake where \texttt{[name]} refers to an optional component, e.g.
\begin{listing}[mybash]
cmake -D BUILD_gui=OFF -D BUILD_tlu=ON ..
\end{listing}
to disable the GUI but enable additionally the TLU producer and
executables.

The corresponding settings are cached, so that they will be again used
next time CMake is run.

Some of the optional packages and producers include:
\begin{description}

\ttitem{main}
The common library, and some command-line programs that depend on only this library

\ttitem{tlu}
The TLU library, and the command-line programs that depend on it.

\ttitem{gui}
The graphical parts of the DAQ, such as the Run Control and Log Collector.

\end{description}

The producers are stored in the \texttt{./producer} subdirectory and
include: \texttt{altro}, \texttt{altroUSB}, \texttt{depfet},
\texttt{eudrb}, \texttt{fortis}, \texttt{mimoroma}, \texttt{mvd},
\texttt{pixelmanproducer}, and \texttt{taki}. These are
user-contributed producers for specific detectors inside the EUDET
telescope.  They should not be compiled unless needed.

A short description of selected producers:
\begin{description}

\ttitem{producers/eudrb}
The code for accessing EUDRB boards over VME.  Depends
on the vme library which will be automatically built
when \texttt{eudrb} is enabled. This should only be compiled on an
MVME6100 single-board computer, as it is only compatible with the
Tundra Tsi148 VME bridge, and PPC processors.
\end{description}



To install the binaries and the library outside the source tree, you
need to set the \texttt{INSTALL\_PREFIX} option, e.g.
\begin{listing}[mybash]
cmake -D INSTALL_PREFIX=/usr/local ..
\end{listing}
to install the executables into the \texttt{bin} and the library into \texttt{lib} subdirectories of \texttt{/usr/local}.

If you ever need to, you can safely remove all files from the build folder
as it only contains automatically generated files. Just run
\begin{listing}[mybash]
cd build
rm -rf *
\end{listing}
to start from scratch.


\subsection{Compiling}
You should just have to run the command:
\begin{listing}[mybash]
make install
\end{listing}

from the top EUDAQ directory to compile the common library,
along with some command-line programs (the contents of the \texttt{./main/exe} subdirectory).
If other parts are needed, you can specify them as arguments to the
CMake command during the configuration step.

The executable binaries and the common shared library will be installed by default into the
\texttt{bin} and \texttt{lib} directories in the source tree,
respectively. If you would like to install into a different location,
please set the respective parameter during the CMake configuration.

\subsection{Compiling the main library and GUI on Windows using Visual~Studio}

This section gives a short overview on the steps needed to compile the
project under Windows (tested under Windows 7, 32-bit). For a more
detailed introduction to the Windows build system and Visual~Studio
project files see the appendix~\ref{app:compileOnWindows} on
page~\pageref{app:compileOnWindows}.

\begin{itemize}
\item Prerequisites: 
\begin{itemize}
\item Download Qt4 or Qt5:
\item Download and install the pthreads library (pre-build binary from
  \url{ftp://sources.redhat.com/pub/pthreads-win32}) into either
  \texttt{c:\\pthreads-w32} or \texttt{./extern/pthreads-w32}
\item Download Visual Studio Express Desktop (e.g. 2013 Version):
  \url{http://www.microsoft.com/en-us/download/details.aspx?id=40787}
\end{itemize}

\item Start the Visual Studio \emph{Developer Command Prompt} from the
  Start~Menu entries for Visual~Studio (Tools subfolder) which opens a
  \texttt{cmd.exe} session with the necessary environment variables
  already set. Now execute the \texttt{qtenv2.bat} batch file (or similar) in the Qt folder, e.g.
  
  \begin{listing}[mybash]
C:\Qt\Qt5.1.1\5.1.1\msvc2012\bin\qtenv2.bat
\end{listing}
Replace "5.1.1" with the version string of your Qt installation.

\item Now clone the EUDAQ repository (or download using GitHub) and enter the build directory on the prompt, e.g. by entering

  \begin{listing}[mybash]
cd c:\Users\[username]\Documents\GitHub\eudaq\build
\end{listing}

\item Configuration: Now enter

  \begin{listing}[mybash]
cmake ..
\end{listing}

to generate the VS project files.

\item Compile by calling
  \begin{listing}[mybash]
MSBUILD.exe EUDAQ.sln /p:Configuration=Release
\end{listing}
or install into \texttt{eudaq\\bin} by running
  \begin{listing}[mybash]
MSBUILD.exe INSTALL.vcxproj /p:Configuration=Release
\end{listing}
\item This will compile the main library and the GUI; for the remaining processors, please check the individual documentation.
\end{itemize}

Note on ``\emph{moc.exe - System Error: The program can't start
  because MSVCP110.dll is missing from your computer}'' errors: when using Visual~Express~2013 and \texttt{pthreads-w32} 2.9.1, you might require ``Visual C++ Redistributable for Visual Studio 2012'': download (either x86 or x64) from \url{http://www.microsoft.com/en-us/download/details.aspx?id=30679} and install.
