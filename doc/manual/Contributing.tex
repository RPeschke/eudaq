% !TEX root = EUDAQUserManual.tex
\section{Developing and Contributing to EUDAQ}

If you would like to contribute your code back into the main repository, please follow the ``fork \& pull request'' strategy:

\begin{itemize}
\item Create a user account on github, log in
\item ``Fork'' the (main) project on github (using the button on the page of the main repo)
\item \emph{Either}: clone from the newly forked project and add
  'upstream' repository to local clone (change user names in URLs
  accordingly):
  \begin{listing}[mybash]
git clone https://github.com/hperrey/eudaq eudaq
cd eudaq
git remote add upstream https://github.com/eudaq/eudaq.git
\end{listing}
\item \emph{or} if edits were made to a previous checkout of upstream: rename origin to upstream, add fork as new origin:

  \begin{listing}[mybash]
cd eudaq
git remote rename origin upstream
git remote add origin https://github.com/hperrey/eudaq
git remote -v show
\end{listing}
\item Optional: edit away on your local clone!
\item Push the edits to origin (our fork)
  \begin{listing}[mybash]
git push origin
\end{listing}
(defaults to \texttt{git push origin master} where origin is the repo and master the branch)
\item Verify that your changes made it to your github fork and then click there on the ``compare \& pull request'' button
\item Summarize your changes and click on ``send''
\item Thank you!
\end{itemize}
